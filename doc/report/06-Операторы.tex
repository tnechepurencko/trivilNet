\hypertarget{statements}{%
\section{Операторы}\label{stmt:chapter}}

Операторы задают действия.

\begin{Grammar}
Оператор
    : Локальное-описание
    | Простой-оператор
    | Оператор-если
    | Оператор-надо
    | Оператор-когда
    | Оператор-пока
    | Оператор-вернуть
    | Оператор-прервать
    | Оператор-авария

Простой-оператор
    : Оператор-выражение 
    | Оператор-присваивания
    | Инкремент 
    | Декремент
\end{Grammar}

\hypertarget{blocks}{%
\subsection{Блоки}\label{stmt:blocks}}

Операторы сгрупированы в \emph{Блоки}, которые задают последовательность действий для функций и методов (\See{decls:functions}, \See{decls:methods}) и входа в модуль (\See{mods:entry}).
Блок - это, возможно, пустая последовательность операторов, которые могут включать локальные описания.


\begin{Grammar}
Блок: '{' Список-операторов? '}'
Список-операторов: Оператор (Разделитель Оператор)* 
\end{Grammar}

\hypertarget{local-decls}{%
\subsection{Локальные описания}\label{stmt:local-decls}}

Локальное описание определяет идентификатор, для которого областью действия (\See{decls:scopes}) является часть блока, 
от точки завершения описания до завершающей скобки блока, исключая вложенные блоки, в которых описан такой же идентификатор.

\begin{Grammar}
Локальное-описание: Описание-переменной
\end{Grammar}

Для локальных переменных не может быть задана \emph{поздняя инициализация} (\See{decls:variables}).

\TBD: Нужны ли локальные константы? Или это задача оптимизации?

\hypertarget{expr-stmt}{%
\subsection{Выражение, как оператор}\label{stmt:expr-stmt}}

Выражение, являющееся вызовом функции или метода, кроме вызова стандартных функций (\See{stdfuncs:std-funcs}), может быть использовано как оператор. 
Результат вызова при этом игнорируется. 

\begin{Grammar}
Оператор-выражение: Выражение
\end{Grammar}

\begin{Trivil}
напечатать("Привет")
Факториал(5)
\end{Trivil}

Не могут быть использованы в качестве оператора:
\begin{SampleErr}[vspace=2pt]
1 + 1
длина(а)
\end{SampleErr}

\hypertarget{assignment}{%
\subsection{Оператор присваивания}\label{stmt:assignment}}

Присваивание заменяет текущее значение, хранящееся в переменной, новым значением, заданным выражением.

\begin{Grammar}
Оператор-присваивания: Выражение ':=' Выражение
\end{Grammar}

Выражение в левой части должно быть \emph{изменяемым}:

\smallskip
\begin{tabular}[l]{l|l|p{4.8cm}}
 \multicolumn{2}{l|}{\textbf{Левое выражение}}   & \textbf{Условие}  \\ 
\hline
Имя объекта  & \verb+пер+ или \verb+мод.пер+ &  \emph{пер} - это изменяемая переменная (\See{decls:variables}) \\
Доступ к полю & \verb+что-то.поле+ & \emph{что-то} - это изменяемое выражение, \emph{поле} - это изменяемое поле (\See{decls:fields-init}) \\
Индексация & \verb+что-то[индекс]+ & \emph{что-то} - это изменяемое выражение \\
Преобразование & \verb+что-то(:Тип)+ & \emph{что-то} - это изменяемое выражение \\
\hline
\end{tabular}

\bigskip
Выражение в правой части должно быть \emph{совместимо по присваиванию} с типом левого выражение (\See{compat:assign}).

\begin{Trivil}
пусть ц := 0
пусть байты := Байты[1, 2, 3]
ц := 2
байты[ц] = 5
\end{Trivil}

Ошибка присваивания в неизменяемое выражение:
\begin{SampleErr}[vspace=2pt]
пусть ц = 0
ц := 2
\end{SampleErr}

\hypertarget{inc-dec}{%
\subsection{Инкремент и декремент}\label{stmt:inc-dec}}

Операторы \emph{Инкремент} ('++') и \emph{Декремент} ('--') увеличивают или уменьшают свои операнды на 1. 

\begin{Grammar}
Инкремент: Выражение '++'
Декремент: Выражение '--'
\end{Grammar}

Как и в случае присваивания (\See{stmt:assignment}), выражение должно быть \emph{изменяемым}.
Тип выражения должен быть одного из типов: Байт, Цел64, Слово64.

\begin{Trivil}
пусть ц := 1
пусть байты := Байты[1, 2, 3]
байты[ц]++
\end{Trivil}

\hypertarget{if-stmt}{%
\subsection{Оператор 'если'}\label{stmt:if-stmt}}

Оператор \keyword{если} определяет условное выполнение двух ветвей в соответствии со значением логического выражения. 
Если выражение принимает значение \verb+истина+, выполняется Блок первой веткви, в противном случае, выполняется ветвь \keyword{иначе}, если она задана.

\begin{Grammar}
Оператор-если: 
    'если' Выражение Блок ('иначе' (Оператор-если | Блок))?
\end{Grammar}

\begin{Trivil}
если ц > макс { ц := макс }

если Буква?(сим) { Имя() }
иначе если Цифра?(сим) { Число() }
иначе { Ошибка!() }
\end{Trivil}

\hypertarget{guard-stmt}{%
\subsection{Оператор 'надо'}\label{stmt:if-guard}}

Оператор \keyword{надо} используется для завершения исполнения операторов некоторого контекста, 
если условие, заданое логическим выражением, не выполнено.

\begin{Grammar}
Оператор-надо: 
    'надо' Выражение 'иначе' (Завершающий-оператор | Блок)
Завершающий-оператор
    : Оператор-вернуть
    | Оператор-прервать
    | Оператор-авария
\end{Grammar}

Если в ветви \keyword{иначе} стоит \emph{Блок}, то последним оператором блока должен быть завершающий оператор.

\smallskip
\begin{tabular}[c]{r|p{6.5cm}}
 \textbf{Завершающий оператор} & \textbf{Действие}  \\ 
\hline
\keyword{вернуть} (\See{stmt:return-stmt}) & выход из тела функции, метода или входа \\
\keyword{прервать} (\See{stmt:break-stmt}) & выход из ближайшего объемлющего цикла \\
\keyword{авария} (\See{stmt:crash-stmt}) & аварийное завершение программы \\
\hline
\end{tabular}

\begin{Trivil}
надо делитель # 0 иначе авария("деление на ноль")

надо число > 1 иначе вернуть 1
\end{Trivil}

\hypertarget{when-stmt}{%
\subsection{Оператор 'когда' или оператор выбора}\label{stmt:when-stmt}}

Оператор \keyword{когда} выбирает одну ветвь выполнения из нескольких вариантов, в соответствии со значением выражения.

\bigskip
\TBD: Добавить предикатный и оператор выбора по типу?

\begin{Grammar}
Оператор-когда: 
    'когда' Выражение '{'
    Вариант*
    ('иначе' Список-операторов?)?
    '}'
Вариант:
    'есть' Выражение (',' Выражение)* ':' Список-операторов?
\end{Grammar}

При выполнении оператора \keyword{когда} вычисляется \emph{Выражение},
затем выражения в \emph{Вариантах} вычисляются слева направо и сверху вниз. 
Как только значение выражения \emph{Варианта} стало равным значению первого выражения, выполняется список операторов этого \emph{Варианта}.
Остальные варинты игнорируются.
Если ни один из вариантов не выполнен, и есть ветка \keyword{иначе}, выполняется список операторов этой ветки.

Тип каждого выражения в вариантах должен быть равен типу первого выражения (\See{compat:equal-types}).

\begin{Trivil}
когда х {
есть 0: вернуть "ничего"
есть 1: вернуть "один"
есть 2: вернуть "два"
иначе вернуть "много"
}
\end{Trivil}


\hypertarget{while-stmt}{%
\subsection{Оператор цикла 'пока' }\label{stmt:while-stmt}}

Оператор \keyword{пока} определяет повторное выполнение блока до тех пор, пока значение логического выражение равно \verb+истина+. 
Выражение вычисляется перед каждой итерацией.

\begin{Grammar}
Оператор-пока: 'пока' Выражение Блок 
\end{Grammar}

\begin{Trivil}
пока а < б {
    а := а * 2
}
\end{Trivil}

\hypertarget{break-stmt}{%
\subsection{Оператор 'прервать'}\label{stmt:break-stmt}}

Оператор \keyword{прервать} завершает выполнение самого внутреннего оператора цикла в рамках одной и той же функции.

\begin{Grammar}
Оператор-прервать: 'прервать' 
\end{Grammar}

\begin{Trivil}
пока истина {
    а := а * 2
    если а > б { прервать }
}
\end{Trivil}

\hypertarget{return-stmt}{%
\subsection{Оператор 'вернуть'}\label{stmt:return-stmt}}

Оператор \keyword{вернуть} завершает выполнение тела функции, метода или входа, и, возможно, возращает значение.

\begin{Grammar}
Оператор-вернуть: 'вернуть'  (Выражение | Разделитель)
\end{Grammar}

Если оператор \keyword{вернуть} используется в теле входа или в теле функции или метода, в сигнатуре которых не указан тип результата, 
то выражение в операторе должно отсутствовать.

И наоборот, если в оператор используется в теле функции или метода, в сигнатуре которых указан тип результата, выражение должно быть и 
оно должно быть \emph{совместимо по присваиванию} с типом результата (\See{compat:assign}).

\begin{Trivil}
фн Факториал(ц: Цел64): Цел64 {
    если ц <= 1 { вернуть 1 }
    вернуть ц * Факториал(ц - 1)
}
\end{Trivil}

\hypertarget{crash-stmt}{%
\subsection{Оператор 'авария' и аварийное завершение}\label{stmt:crash-stmt}}

Оператор \keyword{авария} запускает аварийную ситуация, которая, обычно, приводит к аварийному завершению программы. 

\begin{Grammar}
Оператор-вернуть: 'авария' '(' Выражение ')'
\end{Grammar}

Тип выражения должен быть Строка. Значение этого выражения используется в сообщении об аварийной ситуации.

Кроме явного запуска аварийной ситуации, аварийная ситуация может быть запущена неявно, в следующих случаях:
\begin{d_itemize}
\item
    Выход индекса за пределы вектора при индексации
\item
    Операция подтверждения типа выполнена над объектом со значением 'пусто'
\item
    Недопустимое преобразования типа 
\item
    Недопустимая последовательность байт в кодировке UTF-8
\item
    Невозможность выделения памяти для динамического объекта
\item
    Нереализованная возможность в системе поддержки выполнения
\end{d_itemize}

\bigskip
Язык \thelang{} не содержит средств перехвата аварийной ситуации или восстановления после него. 
Такие средства могут быть добавлены на уровне библиотек.
