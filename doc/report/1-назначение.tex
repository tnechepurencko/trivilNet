\hypertarget{target}{
\section{Назначение}\label{target:chapter}}
%\hypertarget{expressions}{%
%\section{Expressions}\label{expressions}}

Язык программирования \thelang{} разработан в рамках работы на семейством языков программирования \href{http://алексейнедоря.рф/?p=419}{\myref{Языки выходного дня}}  (ЯВД)  проекта \href{http://digital-economy.ru/stati/интенсивное-программирование}{\myref{Интенсивное программирования}}.

\thelang{} является нулевым языком семейства ЯВД, предназначенным для реализации компиляторов и экосистемы других языков семейства. В рамках классификации языков, принятом в проекте Интенсивное программирования, это язык L2.


Основными требованиями к языку при разработке были поставлены
\begin{d_itemize}
\item 
    Язык должен быть минимально достаточным для удобной разработки компиляторов. Требование это во многом субъективно, так как компиляторы можно писать существенно по разному.
\item 
    Язык должен быть русскоязычным и с синтаксисом минимизирующим переключение на латиницу в процессе разработки программ.
\end{d_itemize}

Название языка происходит от слова "тривиальный", что означает, что при разработке языка практически везде использовались решения, проверенные в других современных языках программирования, в первую очередь "донорами" являются Go, Swift, Kotlin и Oberon.

Несмотря на узкую направленность на разработку компиляторов, \thelang{} является языком программирования общего назначения, пригодным для решения широкого круга задач.

Язык (и экосистема) обладает существенными предпосылками для использование его в качестве учебного языка для обучения студентов разработке компиляторов, библиотек, средств разработки, алгоритмов оптимизации и так далее, в первую очередь это:
\begin{d_itemize}
\item 
    Простота языка
\item 
    Современный вид и набор конструкций языка
\item 
    Простота компилятора
\item 
    Открытая лицензия.
\end{d_itemize}

