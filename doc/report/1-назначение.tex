\hypertarget{назначение}{
\section{Назначение}\label{наз:назначение}}
%\hypertarget{expressions}{%
%\section{Expressions}\label{expressions}}

Язык программирования \thelang{} разработан в рамках подпроекта Языки выходного дня  \needlink{} проекта Интенсивное программирования \needlink{}.

\thelang{} является нулевым языком семейства Языки выходного дня, предназначенным для реализации компиляторов и экосистемы других языков семейства. В рамках классификации языков, принятом в проекте Интенсивное программирования, это язык L2.

Основными требованиями к языку при разработке были поставлены
\begin{itemize}
\item 
    Язык должен быть минимально достаточным для удобной разработки компиляторов. Требование это во многом субъективно, так как компиляторы можно писать существенно по разному.
\item 
    Язык должен быть построен на основе русского языка с минимизацией переключение между языками в процессе разработки программ на нем.
\end{itemize}

Название языка происходит от слова "тривиальный", что означает, что при разработке языка практически везде использовались решения, проверенные в других современных языках программирования, в первую очередь "донорами" языка являются Go, Swift, Kotlin и Oberon.

Несмотря на узкую направленность на разработку компиляторов, \thelang{} язык является языком программирования общего назначения, пригодным для решения широкого круга задач.

Язык (и экосистема) обладает существенными предпосылками для использование его в качестве учебного языка для обучения студентов разработке компиляторов, библиотек, средств разработки, алгоритмов оптимизации и так далее, в первую очередь это:
\begin{itemize}
\item 
    Простота языка.
\item 
    Современность языка.
\item 
    Простота компилятора (первый компилятор написан на Go, второй будет написан на Тривиле).
\item 
    Открытая лицензия.
\end{itemize}

