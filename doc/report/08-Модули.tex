\hypertarget{modules}{%
\section{Модули}\label{mods:chapter}}

Программа на языке \thelang{} состоит из модулей (единиц компиляции), исходный текст каждого модуля расположен в одном или нескольких исходных файлов.

Каждый исходный файл состоит из заголовка модуля, за которым следует, возможно, пустой список импорта, 
за которым следует, возможно, пустой набор описаний типов, констант, переменных, функций и методов и, возможно, вход в модуль.

\begin{Grammar}
Модуль: Исходный-файл+
Исходный-файл:
    Заголовок-модуля
    Список-импорта
    Описание-или-вход*

Заголовок-модуля:
    'модуль' Идентификатор Разделитель
     ('осторожно' Разделитель)?

Описание-или-вход: Описание | Вход
\end{Grammar} 

\hypertarget{mod-header}{%
\subsection{Заголовок модуля}\label{mods:mod-header}}

Каждый исходный файл начинается с заголовка модуля. Заголовок содержит идентификтор, который определяет модуль,
к которому принадлежит файл. Идентификатор не принадлежит никакой области действия и не связан ни с каким объектом.

Заголовок модуля может содержать признак \keyword{осторожно}, который означает, что в исходном файле можно использовать 
\emph{небезопасные операции} (\See{unsafe:chapter}).

\hypertarget{import}{%
\subsection{Импорт}\label{mods:import}}

Наличие импорта в исходном файле указывает, что этот исходный файл зависит от функциональности импортируемого модуля. Импорт обеспечивает доступ к экспортированным идентификаторам импортируемого модуля.

Каждый импорт содержит путь импорта, указывающий на размещение импортируемого модуля.

\begin{Grammar}
Список-импорта: Импорт*
Импорт: 'импорт' Путь-импорта Разделитель
Путь-импорта: '"' Путь-в-хранилище | Файловый-путь '"' 
\end{Grammar} 

\TBD: явное задание имени импорта.

\bigskip
Импорт добавляет описание идентификатора, текстуально равного последнему имени в пути импорта, который будет использоваться для доступа к экспортированным объектам импортируемого модуля. 

\begin{Trivil}
модуль x

импорт "стд::вывод"

вход {
    вывод.ф("Привет!")
}
\end{Trivil}

Путь импорта может быть задан как \emph{путь в хранилище} исходных текстов или как \emph{файловый путь}. 
В любом случае, в итоге, он должен указывать на папку, содержащую исходные файлы импортируемого модуля.

\begin{Grammar}[vspace=2pt]
Путь-в-хранилище: 
    Имя-хранилища '::' Имя-папки ('/' Имя-папки)*
\end{Grammar} 

\emph{Путь в хранилище} состоит из имени хранилища, которое явным или неявным образом (см. README) должно быть привязано к определенной папки файловой системы и пути, относительно этой папки:
\begin{Trivil}[vspace=2pt]
импорт "стд::юникод/utf8" 

пусть декодер = utf8.декодер(...)
\end{Trivil}

Язык не определяет формально набор символов, которые могут использоваться в  \emph{имени хранилища} и  \emph{имени папки}, надеясь на здравый смысл разработчиков.

\emph{Файловый путь} может быть аболютным или относительным (в терминах инструментальной платформы). 
Не рекомендуется использовать абсолютные пути, так как это приводит к непереносимому коду. 
Относительный путь всегда трактуется как путь относительно той папки, в которой запущен компилятор (рабочая папка). 

Например в случае импорта:
\begin{Trivil}[vspace=2pt]
импорт "трик/лексер"
\end{Trivil}
будет импортирован модуль, который расположен в подпапке 'трик/лексер' рабочей папки.

Имена папок должны разделяться символом \verb+/+ (слеш) независимо от инструментальной платформы, на которой используется компилятор.

%Ошибка компиляции, если знак экспорта указан для идентификатора, описанного не на уровне модуля.

\hypertarget{entry}{%
\subsection{Вход или инициализация модуля}\label{mods:entry}}

Модуль может содержать действия, которые выполняются при инициализации модуля - \emph{вход в модуль}. 

\begin{Trivil}
Вход: 'вход' Блок
\end{Trivil}

\TBD: описать: переменные с поздней инициализацией
\bigskip

Система выполняет должна обеспечивать следующие условия инициализации модуля М:
\begin{d_itemize}
\item
    инициализация модуля выполняется один раз
\item
    инициализации М выполняется \textbf{после} инициализации всех модулей, которые импортирует М
\item
    инициализации М выполняется \textbf{до} инициализации тех модулей, которые импортируют М
\end{d_itemize}

\hypertarget{execution}{%
\subsection{Инициализация и исполнение программы}\label{mods:execution}}

Программа состоит из головного модуля и всех прямо или косвенно импортированных модулей.

Исполнение программы состоит из:
\begin{d_itemize}
\item
    Инициализации всех модулей, импортированных из головного модуля. Это приводит к рекурсивной инициализации всех используемых модулей
    программы. Для корректной инициализации граф импорта должен быть ациклическим.
\item
    И, затем, выполнение входа головного модуля.
\end{d_itemize}

