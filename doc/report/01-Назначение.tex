\hypertarget{target}{
\section{Назначение}\label{target:chapter}}
%\hypertarget{expressions}{%
%\section{Expressions}\label{expressions}}

Язык программирования \thelang{} разработан в рамках работы на семейством языков программирования \href{http://алексейнедоря.рф/?p=419}{\myref{Языки выходного дня}}  (ЯВД)  проекта \href{http://digital-economy.ru/stati/интенсивное-программирование}{\myref{Интенсивное программирования}}.

\thelang{} является предварительным языком семейства ЯВД, предназначенным для реализации компиляторов и экосистемы следующих языков семейства. В рамках классификации языков, принятом в проекте Интенсивное программирования, это язык L2.

Основными требованиями к языку были поставлены:
\begin{d_itemize}
\item 
    язык должен быть минимально достаточным, то есть в него должны быть включены только те типы и конструкции, которые необходимы для реализации компиляторов
\item 
    язык должен быть современным с точки зрения набора конструкций
\item 
    язык должен поддерживать надежное программирование (автоматическое управление памятью, отсутствие неопределенного поведение, безопасность указателей, 
    минимизация неявных конструкций)
\item 
    язык должен обеспечивать легкость чтения и понимания (readability) и легкость разработки
\item 
    как следствие: язык должен быть русскоязычным. Лексика и синтаксис языка должны минимизировать переключение на латиницу и обратно в процессе разработки программ
\end{d_itemize}

Название языка происходит от слова "тривиальный", что означает, что при разработке языка практически везде использовались решения, проверенные в других современных языках программирования, в первую очередь "донорами" являются Go, Swift, Kotlin и Oberon.

В итоге разработки, несмотря на первоочередную направленность на разработку компиляторов, \thelang{} является
языком программирования общего назначения, пригодным для решения широкого круга задач.
 
\bigskip
Получившийся язык (и экосистема) обладает существенными достоинствами для использования в качестве полигона для обучения студентов 
разработке компиляторов, библиотек, алгоритмов оптимизации и так далее, в первую очередь это:
\begin{d_itemize}
\item 
    Простота языка
\item 
    Современный вид и набор конструкций языка
\item 
    Простота компилятора
\item 
    Открытая лицензия
\end{d_itemize}

