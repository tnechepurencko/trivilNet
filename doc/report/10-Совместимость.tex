\hypertarget{compat}{%
\section{Правила совместимости}\label{compat:chapter}}

\hypertarget{equal-types}{%
\subsection{Эквивалентность типов}\label{compat:equal-types}}

Во многих случаях, например, для большинства бинарных операция или при переопределения методов, требуется эквивалентность типов. 

Два типа  \emph{Т1} и \emph{Т2} эквивалентны, если выполняется одно из условий:
\begin{d_itemize}
\item
    типы \emph{Т1} и \emph{Т2} это один и тот же тип, возможно, переименованный, так как переименование типа не создает новый тип
\item
    типы \emph{Т1} и \emph{Т2} это типы векторов и типы их элементов эквивалентны
\item
    типы \emph{Т1} и \emph{Т2} это \emph{может быть} типы и их базовые типы эквивалентны
\end{d_itemize}

Переименование типа не создает новый тип, таким образом, после описания: 
\begin{Trivil}
тип Числа = Цел64 
пусть ч: Число = 1
пусть ц: Цел64 = 2
пусть с: Слово64 = 3
\end{Trivil} 
типы переменных \verb+ч+ и \verb+ц+ эквивалентны, так как это один и тот же тип. В то же время, типы переменных \verb+с+ и \verb+ц+ не эквивалентны, так как это разные типы.

\hypertarget{compat-assign}{%
\subsection{Совместимость по присваиванию}\label{compat:assign}}

Совместимость по присваиванию используется в следующих случаях:
\begin{d_itemize}
\item
    инициализация констант и переменных с явным типом: \verb+пусть х: Т = выражение+, выражение должно быть совместимо с типом \verb+Т+
\item
    оператор присваивания: \verb+х := выражение+
\item
    оператор \verb+вернуть выражение+, выражение должно быть совместимо с типом результата функции
\item
    передача фактических параметров при вызове: \verb+Ф(аргумент1, ..., аргументN)+, аргумент должен быть совместим с типом параметра
\end{d_itemize}

Значение типа \emph{ТВ} совместимо по присваиванию с целевым типом \emph{ТП} (типом переменной, константы или параметра),
если выполняется одно из условий:
\begin{d_itemize}
\item 
    \emph{ТП} и \emph{ТВ} эквивалентны
\item 
    \emph{ТП} это целый тип (Байт, Цел64, Слово64), а значение - это литерал целого типа, причем значение литерала входит в допустимый диапазон значений \emph{ТП} 
\item 
    \emph{ТП} и \emph{ТВ} это типы векторов и типы их элементов эквивалентны
\item 
    \emph{ТП} это \emph{\keyword{мб} Т1}, а значение есть 'пусто' или \emph{ТП} это \emph{\keyword{мб} Т2} и типы \emph{Т1} и \emph{Т2} эквивалентны
\item 
    \emph{ТП} и \emph{ТВ} это типы классов и \emph{ТВ} является расширением  \emph{ТП}
\item 
    переменная является полиморфным параметром
\end{d_itemize}




