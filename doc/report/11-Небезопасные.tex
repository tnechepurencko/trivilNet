\hypertarget{unsafe}{%
\section{Применять с осторожностью}\label{unsafe:chapter}}

Низкоуровневые типы и операции.

\hypertarget{unsafe-conversions}{%
\subsection{Осторожные преобразования типа}\label{unsafe:conversions}}

Часть преобразований типа являются низкоуровневыми и/или требуют особого внимания разработчика. 
Для использования таких преобразований операция преобразования должна быть помечана ключевым словом \cautiously{осторожно}, 
кроме того исходный файл, в котором используется \emph{небезопасное преобразование} тоже должен быть помечен  ключевым словом \cautiously{осторожно} (\See{mods:chapter}).

\begin{Grammar}
Преобразование: '(:' 'осторожно'? Указ-типа ')'
\end{Grammar}   

Разрешенные преобразования:

\smallskip
\begin{tabular}[c]{r|l}
\textbf{Целевой тип} & \textbf{Тип выражения}  \\ 
\hline
Цел64 & Слово64  \\
Вещ64 & Слово64  \\
Слово64 & Цел64  \\
Слово64 & Вещ64  \\
Слово64 & ссылочный тип  \\
ссылочный тип & Слово64 \\
Строка & Строка8  \\
Строка8 & Строка  \\
\hline
\end{tabular}

\bigskip
Ссылочными типами являются Строка, типы вектора и типы класса.

Все преобразования выполняются  без изменения битового представления. 
Для преобразования Слово64 в ссылочный тип выполняется проверка времени исполнения, 
и запускается \emph{аварийная ситуация} (\See{stmt:crash-stmt}) в случае несовпадения целевого типа и динамического типа выражения.

\hypertarget{type-string8}{%
\subsection{Тип Строка8}\label{unsafe:type-string8}}

Тип Строка8 введен для удобной и быстрой работы со строками в стандартных библиотеках. 
Единственным способом получения значения типа Строка8 является осторожное преобразования из строки. 
Это преобразование выполняется только во время компиляции.

\begin{Trivil}
пусть с8 = "Привет"(:осторожно Строка8)
\end{Trivil}

Значение типа Строка8 - это неизменяемая, индексируемая последовательность байтов.
Стандартная функция \verb+длина+ примененная к значению типа Строка8 возвращает число байтов.

\bigskip
Сравнение типов:

\bigskip
\begin{tabular}[c]{l|p{3.5cm}|p{3.5cm}}
& \textbf{Строка} & \textbf{Строка8}  \\ 
\hline
Определение &  последовательность символов & последовательность байтов \\
Индексация & нет & да \\
\verb+длина+(c) & число символов & число байтов \\
можно менять & нет & нет \\
\hline
\end{tabular}

\begin{Trivil}
пусть с = "Привет"
пусть с8 = с(:осторожно Строка8)

вывод.ф("%v\n", длина(с)) // выведет: 6
вывод.ф("%v\n", длина(с8)) // выведет: 12
вывод.ф("%v\n", с8[0]) // выведет: 0xD0
\end{Trivil}

Значение типа Строка8 неизменяемое, менять байты в нем нельзя:
\begin{SampleErr}[vspace=2pt]
    пусть с8 = "Привет"(:осторожно Строка8)
    с8[0] := 1 // ошибка
\end{SampleErr}