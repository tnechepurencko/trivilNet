\hypertarget{unsafe}{%
\section{Применять с осторожностью}\label{unsafe:chapter}}

Низкоуровневые типы и операции.

\hypertarget{unsafe-conversions}{%
\subsection{Осторожные преобразования типа}\label{unsafe:conversions}}

Часть преобразований типа являются низкоуровневыми и/или требуют особого внимания разработчика. 
Для использования таких преобразований операция преобразования должна быть помечана ключевым словом \cautiously{осторожно}, 
кроме того исходный файл, в котором используется \emph{небезопасное преобразование} тоже должен быть помечен  ключевым словом \cautiously{осторожно} (\See{mods:chapter}).

\begin{Grammar}
Преобразование: '(:' 'осторожно'? Указ-типа ')'
\end{Grammar}   

Разрешенные преобразования:

\smallskip
\begin{tabular}[c]{r|l}
\textbf{Целевой тип} & \textbf{Тип выражения}  \\ 
\hline
Цел64 & Слово64  \\
Вещ64 & Слово64  \\
Слово64 & Цел64  \\
Слово64 & Вещ64  \\
Слово64 & ссылочный тип  \\
ссылочный тип & Слово64 \\
\hline
\end{tabular}

\bigskip
Ссылочными типами являются Строка, типы вектора и типы класса.

Все преобразования выполняются  без изменения битового представления. 
Для преобразования Слово64 в ссылочный тип выполняется проверка времени исполнения, 
и запускается \emph{аварийная ситуация} (\See{stmt:crash-stmt}) в случае несовпадения целевого типа и динамического типа выражения.
