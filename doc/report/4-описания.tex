\hypertarget{decls}{%
\section{Описания и области действия}\label{decls:chapter}}

Каждый идентификатор, встречающийся в программе, должен быть описан, если только это
не предопределенный идентификатор (\See{decls:predefined-idents}). Идентификатор может быть описан как тип, константа, переменная, функция или как поле класса.

\begin{Grammar}
Описание
	: Описание-типов
	| Описание-констант
	| Описание-переменных
	| Описание-функций
\end{Grammar} 

Описанный идентификатор используется для ссылки на связанный объект, в тех частях программы, которые попадают в \emph{область действия} описания. 
Идентификатор не может обозначать более одного объекта в пределах заданной области действия. 
Область действия может содержать внутри себя другие области действия, в которых идентификатор может быть переопределен.

Область действия, которая содержит в себе все исходные тексты на языке \thelang{} называется \emph{Универсум}.

Области видимости:
\begin{itemize}
\item
Областью действия предопределенного идентификатора является Универсум
\item
Областью действия идентификатора, описанного на верхнем уровне (вне какой-либо функции), является весь модуль (\See{mods:chapter}).
\item
Областью действия имени импортируемого модуля является файл (часть модуля), содержащего импорт (\See{mods:import}).
\item
Областью действия идентификатора, обозначающего параметр функции, является тело функции (\See{decls:functions}).
\item
Областью действия идентификатора, описанного в теле функции (\See{decls:functions}) или теле входа (\See{mods:entry}), является часть \emph{блока} (\See{stmt:blocks}), в котором описан идентификатор, от точки завершения описания и до завершения этого блока. 
\end{itemize}

В описании сразу за идентификатором может следовать признак экспорта '*', указывающий, что идентификатор \emph{экспортирован} и может использоваться в другом модуле, \emph{импортирующем} данный (\See{mods:import}). 

\begin{Grammar}
Идент-оп:  Идентификатор '*'?
\end{Grammar} 

\hypertarget{predefined-idents}{%
\subsection{Предопределенные идентификаторы}\label{decls:predefined-idents}}

Следующие идентификаторы неявно описаны в области действия \emph{Универсум}.
\bigskip

Типы (\See{decls:predefined-types}):
\begin{Verbatim}
    Байт Цел64 Слово64 Вещ64 Лог Символ Строка
\end{Verbatim}

Константы типа Лог (\See{decls:predefined-types}):
\begin{Verbatim}
    ложь истина
\end{Verbatim}

Литерал nullable (\See{}):
\begin{Verbatim}
    пусто
\end{Verbatim}

Стандартные функции (\See{stdfuncs:chapter}):
\begin{Verbatim}
    длина тег нечто
\end{Verbatim}

Кроме того, для векторных типов определен набор встроенных методов (\See{stdfuncs:stdvector}).

\hypertarget{predefined-types}{%
\subsection{Предопределенные типы}\label{decls:predefined-types}}

Следующие типы обозначаются предопределенными идентификаторами, значениями данных типов являются:

\smallskip
\begin{tabular}[c]{l|l}
\textbf{Тип} & \textbf{Множество значений} \\ \hline
Байт &  множество целых числа от 0 до 255   \\
Цел64 & множество всех 64-битных знаковых целых \\
Слово64 & множество всех 64-битных беззнаковых целых \\ 
Вещ64 & множество всех 64-разрядных чисел с плавающей запятой стандарта IEEE-754 \\ 
Лог & константы \verb|ложь| и \verb|истина| \\ 
Символ & множество всех Unicode символов \\ 
Строка & множество всех строковых литералов.
\end{tabular}

\bigskip
Операции над значениями этих типов определены в (\See{expr:operators}).

\hypertarget{type-ref}{%
\subsection{Указание типа}\label{decls:type-ref}}

\thelang{} является языком со статической типизацией, что означает, что тип любого объекта языка явно или неявно указывается во время описания объекта. 
Неявное указание типа может быть использовано в описании констант (\See{decls:constants}), переменных (\See{decls:constants}) и полей класса ().

Для явного указании типа используется имя типа, перед которым может стоять ключевое слово \keyword{мб} (\emph{может быть}). 

\begin{Grammar}
Указ-типа: 'мб'? Квалидент
Квалидент: Идентификатор ('.' Идентификатор)?
\end{Grammar} 

Множество значений объекта с типом \keyword{мб} T состоит из значения, обозначенного предопределенным идентификатором  \keyword{пусто} и значений типа Т. 
Тип такого объекта называется \emph{может быть Т} (\See{decls:mb-types}). В англоязычных языках программирования используется термин \emph{nullable type}.


\hypertarget{constants}{%
\subsection{Описание констант}\label{decls:constants}}

Описание константы связывает идентификатор с постоянным значением. 
Значение константы может быть задано явно или неявно, в случае группового описания констант.

\begin{Grammar}
Описание-констант: 'конст' (Константа | Группа-констант)
Константа: Идент-оп (':' Указ-типа)? '=' Выражение
\end{Grammar} 

Если тип константы не указан, то он устанавливается равным типу выражения (\See{expr:chapter}).
Выражение для константы должно вычисляться во время компиляции (\See{expr:const-expr}).

\begin{Trivil}
конст к1: Цел64 = 1 // тип Цел64
конст к2: Байт = 2 // тип Байт
конст к3 = 3 // тип Цел64
конст к4 = "Привет" // тип Строка
\end{Trivil}

Групповое описание констант позволяет опускать тип и выражение для всех констант, кроме первой константы в группе и указать признак экспорт для всех констант группы.

\begin{Grammar}
Группа-констант: 
    '*'? '(' 
    Константа (Разделитель След-константа)* 
    ')'
След-константа: Идент-оп ((':' Указ-типа)? '=' Выражение)?
\end{Grammar} 

Пример группового экспорта:
\begin{Trivil}[vspace=2pt]
конст *(
    Счетчик = 1
    Имя = "Вася"
)
\end{Trivil}

Пример неявного задания значения для констант:
\begin{Trivil}[vspace=2pt]
конст ( 
    / / операции
    ПЛЮС = 1   // тип Цел64, значение = 1
    МИНУС       // тип Цел64, значение = 2  
    ОСТАТОК   // тип Цел64, значение = 3 
    // ключевые слова
    ЕСЛИ = 21   // тип Цел64, значение = 21
    ИНАЧЕ        // тип Цел64, значение = 22
    ПОКА          // тип Цел64, значение = 23
)
\end{Trivil}


\hypertarget{variables}{%
\subsection{Описание переменных}\label{decls:variables}}

\hypertarget{types}{%
\subsection{Описание типов}\label{decls:types}}

\hypertarget{mb-types}{%
\subsection{Может быть типы}\label{decls:mb-types}}


\hypertarget{functions}{%
\subsection{Описание функций}\label{decls:functions}}