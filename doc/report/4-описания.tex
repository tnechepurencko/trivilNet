\hypertarget{decls}{%
\section{Описания и области действия}\label{decls:chapter}}

Каждый идентификатор, встречающийся в программе, должен быть описан, если только это
не предопределенный идентификатор (\See{decls:predefined-idents}).

\begin{Grammar}
Описание
	: Описание-типов
	| Описание-констант
	| Описание-переменных
	| Описание-функций
	;
\end{Grammar} 

Описанный идентификатор используется для ссылки на связанный объект, в тех частях программы, которые попадают в \emph{область действия} описания. 
Идентификатор не может обозначать более одного объекта в пределах заданной области действия. 
Область действия может содержать внутри себя другие области действия, в которых идентификатор может быть переопределен.

Область действия, которая содержит в себе все исходные тексты на языке \thelang{} называется \emph{Универсум}.

Области видимости:
\begin{itemize}
\item
Областью действия предопределенного идентификатора является Универсум
\item
Областью действия идентификатора, описанного на верхнем уровне (вне какой-либо функции), является весь модуль (\See{mods:chapter}).
\item
Областью действия имени импортируемого модуля является файл (часть модуля), содержащего импорт (\See{mods:import}).
\item
Областью действия идентификатора, обозначающего параметр функции, является тело функции (\See{decls:functions}).
\item
Областью действия идентификатора, описанного в теле функции (\See{decls:functions}) или теле входа (\See{mods:entry}), является часть \emph{блока} (\See{stmt:blocks}), в котором описан идентификатор, от точки завершения описания и до завершения этого блока. 
\end{itemize}

Заголовок модуля не является описанием, имя модуля не принадлежит никакой области действия. Его цель - идентифицировать файлы, принадлежащие одному и тому же модулю.

\hypertarget{predefined-idents}{%
\subsection{Предопределенные идентификаторы}\label{decls:predefined-idents}}

\hypertarget{predefined-types}{%
\subsection{Предопределенные типы}\label{decls:predefined-types}}

\hypertarget{constants}{%
\subsection{Описание констант}\label{decls:constants}}

\hypertarget{variables}{%
\subsection{Описание переменных}\label{decls:variables}}

\hypertarget{types}{%
\subsection{Описание типов}\label{decls:types}}

\hypertarget{functions}{%
\subsection{Описание функций}\label{decls:functions}}