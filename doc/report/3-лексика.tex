\hypertarget{lexica}{%
\section{Лексика}\label{lex:chapter}}

Исходный текст есть последовательность лексем: идентификаторов, ключевых слов, литералов, знаков операций и пунктуации. 
Каждая лексема состоит из последовательности Unicode символов (unicode code point) в кодировке UTF-8.

Пробелы (U+0020), символы табуляции (U+0009) и символы завершения строки (U+000D, U+000A) разделяют лексемы, и, игнорируются, кроме следующих случаев:
\begin{itemize}
\item
Символы завершения строк могут использоваться как разделители синтаксических конструкций (\See{lex:separators}).
\item
  Пробелы являются значащими символами в идентификаторах, состоящих из нескольких слов (\See{lex:idents}). 
\end{itemize}

Несколько таких разделителей трактуются, как один.

Исходный текст может содержать \emph{комментарии}.

\hypertarget{comments}{%
\subsection{Комментарии}\label{lex:comments}}

Есть две формы комментариев:
\begin{itemize}
\item
Строчный комментарий начинаeтся с последовательности символов '//' и заканчиваeтся в конце строки.
\item
Блочный комментарий начинается с последовательности символов '/*' и заканчивается последовательностью символов '*/'. 
Блочные комментарии могут быть вложенные.
\end{itemize}

\begin{Grammar}
Комментарий: 
	: '//' (любой символ, кроме завершения строки)*
    | '/*'  ( любой символ)* '*/
	;
\end{Grammar}

\hypertarget{separators}{%
\subsection{Разделители синтаксических конструкций}\label{lex:separators}}

Некоторые синтаксические правила используют нетерминал \emph{Разделитель} для разделения двух подряд идущих синтаксических конструкций, например:
\begin{Grammar}
Список-операторов
	: Оператор (Разделитель Оператор)* 
	;
\end{Grammar}

В качестве разделителя может использоваться символ ';' или символ завершения строки.
\begin{Grammar}
Разделитель
	: ';'
	| символ-завершения-строки
	;	
\end{Grammar}

Пример:
\begin{Trivil}
а := 1; б := 2
в := 1 
\end{Trivil}

В строке 1 операторы разделены символом ';', а оператор в строке 2 отделен от операторов строки 1 символом завершения строки.

Ошибка компиляции - нет разделителя:
\begin{SampleErr}
а := 1 б := 2
\end{SampleErr}

\hypertarget{idents}{%
\subsection{Идентификаторы}\label{lex:idents}}

Идентификатор - это последовательность \emph{слов}, разделенных пробелами или символами дефис '-' с опционально завершающим знаком препинания:

Каждое слово состоит из \emph{Букв} и цифр, и начинается с Буквы. Буквой считается любой Unicode символ, имеющий признак \emph{Letter}, и, дополнительно, символы '№'
и '\_'. 

\begin{Grammar}
Идентификатор
	: Слово ((' ' | '-') Слово)* Знак-препинания?
	;
Слово
	: Буква (Буква | Цифра)*
	;
Буква
	: Unicode-letter
	| '_'
	| '№'
	;
Знак-препинания
	: '?' 
	| '!' 
	;
Цифра
    : '0' .. '9'
    ;
\end{Grammar}

Примеры идентификаторов:
\begin{Trivil}
буква
буква-или-цифра
№-символа
Цифра?
Пора паниковать!
\end{Trivil}
