\hypertarget{lexica}{%
\section{Лексика}\label{lex:chapter}}

Исходный текст есть последовательность лексем: идентификаторов и ключевых слов, литералов, знаков операций, разделителей и комментариев. Каждая лексема состоит из последовательности Unicode символов (unicode code point) в кодировке UTF-8.


\hypertarget{idents}{%
\section{Идентификаторы}\label{lex:idents}}

Идентификатор - это последовательность \emph{слов}, разделенных пробелами или символами дефис '-' с опционально завершающим знаком препинания:

Каждое слово состоит из \emph{Букв} и цифр, и начинается с Буквы. Буквой считается любой Unicode символ, имеющий признак 'Letter', и дополнительно символы '№'
и '\_'. 

\begin{Grammar}
Идентификатор
	: Слово ((' ' | '-') Слово)* Знак-препинания?
	;
Слово
	: Буква (Буква | Цифра)*
	;
Буква
	: Unicode-letter
	| '_'
	| '№'
	;
Знак-препинания
	: '?' 
	| '!' 
	;
Цифра
    : '0' .. '9'
    ;
\end{Grammar}

Примеры идентификаторов:
\begin{Trivil}
буква
буква-или-цифра
№-символа
Цифра?
Пора паниковать!
\end{Trivil}
