\hypertarget{review}{%
\section{Обзор языка}\label{review:chapter}}

\thelang{} - это модульный язык с явным экспортом и импортом, автоматическим управлением памятью (сборка мусора), с поддержкой ООП.

Программа на языке \thelang{} состоит из модулей (единиц компиляции), исходный текст каждого модуля расположен в одном или нескольких исходных файлов.

Пример программы:
\begin{Trivil}
модуль x

импорт "стд/вывод"

вход { 
    вывод.ф("Привет!\n")
}
\end{Trivil}

Для описания языка используется EBNF в формате, близком к формату ANTLR4. Операции: 

\begin{tabular}[c]{r|l}
    () & группировка \\
    X* & повторение 0 и более раз \\
    X+ & повторение 1 и более раз \\
    X? & опциональность X (0 или 1 раз) \\
    X | Y & X или Y \\
\end{tabular}

\bigskip
Пример:
\begin{Grammar}[vspace=0pt]
Список-операторов: Оператор (Разделитель Оператор)* 
\end{Grammar}
